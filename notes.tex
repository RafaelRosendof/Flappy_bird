\documentclass{article}

\title{Rust Game Development Lecture}
\author{Your Name}
\date{\today}

\begin{document}

\maketitle

\section{Introduction}
% Your introduction content goes here.
Aqui vamos para uma introdução e documentação do jogo flappy dragon que vou produzir em rust baseado em um livro para ajudar ao aprendizado da linguagem de programação rust vamos começar com uma seccção que contem a introdução do jogo chamada e ao longo do tempo vou atualizando essa introdução e vou atualizando esse documento para me lembrar de todos os detalhes, além de ter os comentários dentro do código, aqui também vai ter alguns comentários teóricos e muito provável que alguns trechos que eu venha detalhar com mais profundidade. Sem mais delongas vamos ao que interessa, primeira parte 
\vspace{0.3cm}

\section{Introdução}
% Content related to setting up the Rust development environment.
Vamos começar pelo entendimento do loop do jogo, O loop do jogo começa com a abertura da janela seguido dos gráficos e os outros recursos. E normalmente a tela sempre é atualizada entre 30 a 60 vezes por segundo ou fps. Todo passo do jogo que corresponde a esse loop se chama ocasionalmente de "tick()". Todavia vamos resumir para que fique claro. Passo 1 Start, Passo 2 Configure app , Janela e Graphics. Passo 3 Poll OS é para receber o Input State seja o mouse teclado ou joystick, o próprio game engine ou o sistema operacional cuida disso. Passo 4 Chamar a função tick logo em seguida atualizar a tela e por último a opção de Sair ou então volta para o passo 3. 

\\

\vspace{0.2cm}
Nesse projeto vamos utilizar a biblioteca do rust chamada Bracket-lib e Bracket-Terminal, vale olhar a documentação depois incluindo mais exemplos, foi uma biblioteca desenhada para simplificar a curva de aprendizado, fazendo o uso de uma maior abstração em aspectos mais complexos de desenvolvimento de jogos. É uma família de bibliotecas qie incluem geração de números aleatórios, geometria, procurador de arquivos, color handling e alguns algoritimos comuns de desenvolvimento de jogos. 

\\

Bracket-Terminal is the display portion of bracket-lib. Providing an emulated console that con randering in plataforms ranging from text-console to web assembly, including OpenGL, Vulkan and his native suport is in OpenGL. Nota: algumas coisas ficarão em português outras em inglês  

\section{Entendendo o Motor Gráfico no Bracket-lib}
% Abordando conceitos fundamentais de Rust relevantes para o desenvolvimento de jogos.

Nesta seção, vamos explorar o motor gráfico do `bracket-lib` e aprender como utilizá-lo. Primeiramente, vamos criar um projeto com o comando:

\begin{verbatim}
cargo new nome_qualquer
\end{verbatim}

Dentro do arquivo `Cargo.toml`, adicione a seguinte linha abaixo da seção "[dependencies]":

\begin{verbatim}
bracket-lib = "~0.8.1"
\end{verbatim}

Nota: Para auxiliar nesta parte, você pode encontrar um arquivo chamado `figasBrack.rs` que utiliza o `bracket-lib` para fornecer exemplos, incluindo comentários dentro do código.

\impl GameState for State: Esta é uma implementação de um trait chamado `GameState` para a estrutura `State`. Traits em Rust são semelhantes a interfaces em outras linguagens de programação. O trait `GameState` define métodos necessários para que um tipo seja considerado um estado de jogo. No caso, o método `tick` é implementado para atualizar o estado do jogo a cada iteração.

\begin{verbatim}
fn tick(&mut self, ctx: &mut BTerm) { ... }
\end{verbatim}

Este é o método `tick` que faz parte do trait `GameState`. Ele recebe uma referência mutável para a instância de `State` e uma referência mutável para o contexto do jogo (`BTerm`). Este método é chamado a cada iteração do loop principal do jogo.

\begin{verbatim}
ctx.cls();
\end{verbatim}

Limpa o conteúdo do terminal, preparando para a próxima renderização.

\begin{verbatim}
ctx.print(1, 1, "Hello, Bracket Terminal!");
\end{verbatim}

Imprime o texto "Hello, Bracket Terminal!" nas coordenadas (1, 1) do terminal.

\fn main() -> BError { ... }: Esta é a função principal do programa. Ela retorna um tipo `BError`, que representa um erro relacionado à biblioteca `bracket-lib`.

\begin{verbatim}
let context = BTermBuilder::simple80x50() ...
\end{verbatim}

Aqui, é criada uma instância de `BTermBuilder` configurada para um terminal de 80x50 caracteres.

\begin{verbatim}
.with_title("Flappy Dragon") ...
\end{verbatim}

Define o título da janela do terminal como "Flappy Dragon".

\begin{verbatim}
.build()?;
\end{verbatim}

Constrói o contexto do jogo com as configurações especificadas. O `?` é usado para propagar erros, caso a construção falhe.

\begin{verbatim}
main_loop(context, State{});
\end{verbatim}

Inicia o loop principal do jogo chamando a função `main-loop`, passando o contexto do jogo criado e uma instância inicial do estado do jogo (`State{}`).

\\

O uso do \impl{} é para desenvolver métodos que só podem ser acessados pela aquela `struct`. Por exemplo, se definirmos uma `struct` chamada `Retangulo` com altura e largura como floats, podemos criar uma \impl{} para `Retangulo` que contém uma função \fn{} `area(&self)` que retorna um `f64`, multiplicando `self.width * self.height`.

\\

Houve uma pausa no capítulo 3, página 51, abordando erros.


\section{Game Development Basics}
% Introduction to game development concepts using Rust.

\section{Graphics and Rendering}
% Discussing graphics and rendering in Rust game development.

\section{Input Handling}
% Covering input handling techniques for Rust games.

\section{Game Logic and State}
% Exploring game logic and state management in Rust.

\section{Conclusion}
% Summarizing key points and concluding the lecture.

\end{document}

